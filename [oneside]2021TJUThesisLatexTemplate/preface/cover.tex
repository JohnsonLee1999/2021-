% !Mode:: "TeX:UTF-8"

%%  可通过增加或减少 setup/format.tex中的
%%  第274行 \setlength{\@title@width}{8cm}中 8cm 这个参数来 控制封面中下划线的长度。

\cheading{天津大学~2021~届本科生毕业论文}      % 设置正文的页眉,需要填上对应的毕业年份
\ctitle{新型二维材料结构和物性的理论预测}    % 封面用论文标题,自己可手动断行
\caffil{理学院} % 学院名称
\csubject{严济慈物理英才班}   % 专业名称
\cgrade{2017~级}            % 年级
\cauthor{李佩璇}            % 学生姓名
\cnumber{3017210}        % 学生学号
\csupervisor{胡~~杜}        % 导师姓名
\crank{副教授}              % 导师职称

\cdate{\the\year~年~\the\month~月~\the\day~日}

\cabstract{
中文摘要一般在~400~字以内,简要介绍毕业论文的研究目的、方法、结果和结论,语言力求精炼。中英文摘要均要有关键词,一般为~3~—~7~个。字体为小四号宋体,各关键词之间要有分号。英文摘要应与中文摘要相对应,字体为小四号~Times New Roman,详见模板。
}

\ckeywords{关键词~1;关键词~2;关键词~3;……;关键词~7(关键词总共~3~—~7~个,最后一个关键词后面没有标点符号)}

\eabstract{
The upper bound of the number of Chinese characters is 400. The abstract aims at introducing the research purpose, research methods, research results, and research conclusion of graduation thesis, with refining words. Generally speaking, both the Chinese and English abstracts require the keywords, the number of which varies from 3 to 7, with a semicolon between adjacent words. The font of the English Abstract is Times New Roman, with the size of 12pt(small four).
}

\ekeywords{keyword 1, keyword 2, keyword 3, ……, keyword 7 (no punctuation at the end)}

\makecover

\clearpage
